\documentclass[a4paper,11pt]{article}
\usepackage{amsmath}
\usepackage{amssymb}
\usepackage[frenchb]{babel}
\usepackage[utf8]{inputenc}
\usepackage[T1]{fontenc}
\usepackage{array}
\usepackage{minted}
\usemintedstyle{colorful}
\usepackage{xcolor,graphicx}
\usepackage{float}
\usepackage[left=1.5cm,top=0.5cm,right=1.5cm,bottom=0.5cm,noheadfoot]{geometry}
\newenvironment{conditions}
  {\par\vspace{\abovedisplayskip}\noindent\begin{tabular}{>{$}l<{$} @{${}={}$} l}}
  {\end{tabular}\par\vspace{\belowdisplayskip}}
  
\newcommand{\ts}{\textsuperscript}
\setlength{\parindent}{0pt}
\title{Mathématique - TP Fourier}
\author{Ibanez Thomas, Lovino Maxime}
\begin{document}
\maketitle
\section{Introduction}
Pour ce TP il nous a été demandé d'utiliser MatLab pour calculer numériquement des transformées de Fourier (via fft) et leurs inverses (via ifft) et de modifier ces transformée pour créer un filtre.
\section{Questions de l'énoncé}

\subsection{Serie de Fourier de f(t)}
Soit la fonction $f(t) = cos(2\pi t) + 0.9*cos(2\pi 10t)$ \newline
Nous devons calculer les coefficients de la serie de Fourier de cette fonction (rappel, une serie de fourier est calculée par cette formule: 
\begin{equation*}
f(t) = a_0 + \sum_{k=0}^\infty a_k * cos(2\pi f tk) + b_k * sin(2\pi f tk)
\end{equation*}
Où
\begin{conditions}
a_0 &  $\frac{1}{T} \int\limits_{-\frac{T}{2}}^{\frac{T}{2}}  f(t) dt $ \\
a_k &  $\frac{2}{T} \int\limits_{-\frac{T}{2}}^{\frac{T}{2}} f(t) * cos(2\pi ftk) dt$ $ \forall k > 0$\\
b_k & $\frac{2}{T} \int\limits_{-\frac{T}{2}}^{\frac{T}{2}} f(t) * sin(2\pi ftk) dt$ $ \forall k > 0$
\end{conditions}
La periode de cette fonction est de 1 \newline
Analytiquement nous (et wolfram) obtenons
$a_0 = 0$, $a_1 = 1$, $a_{2...9} = 0$, $a_{10} = 0.9$ (tous les coefficient $b_k$ sont nuls)

\subsection{Transformée de f(t)}
La transformée de Fourier de cette fonction calculée numériquement par matlab donne ceci: \newline
\begin{center}
\includegraphics[scale=0.6]{"fhat.png"} \newline
\end{center}
On obtiens donc 2 "diracs" a 1 et 10 Hz ce qui correspond bien au fréquences trouvées dans la serie de Fourier. Les hauteurs sont respectivement 1 et 0.9 ce qui correspond également au coefficients trouvés dans la serie de Fourier.

\subsection{Transformée inverse de $\hat{f}(\omega)$}
Si on fait la transformée inverse de $\hat{f}(\omega)$
On obtient une fonction très proche de l'originale.
\begin{center}
\includegraphics[scale=0.6]{"f_and_f_ifft.png"} \newline
La fonction originale est tracée en bleu, et la fonction recomposée est tracée en rouge. (avec une periode d'échantillonnage de 0.025 [s]) 
\end{center} 

\begin{center}
\includegraphics[scale=0.6]{"f_and_f_ifft1.png"} \newline
La fonction originale est tracée en bleu, et la fonction recomposée est tracée en rouge. (avec une periode d'échantillonnage de 0.05 [s]) 
\end{center} 

\begin{center}
\includegraphics[scale=0.6]{"f_and_f_ifft2.png"} \newline
La fonction originale est tracée en bleu, et la fonction recomposée est tracée en rouge. (avec une periode d'échantillonnage de 0.1 [s]) 
\end{center} 

Moins on a d'échantillons, moins la fft est priécise et donc moins la recomposition est proche de l'originale.

\subsection{Filtre}
Si on filtre toutes les fréquences > 5 Hz on supprime le deuxième pic et on trouve la fonction suivante après ifft.
\begin{center}
\includegraphics[scale=0.6]{"fhatlow.png"} \newline
\end{center}
\begin{center}
\includegraphics[scale=0.6]{"f_and_Fhatlow.png"} \newline
La fonction originale est tracée en bleu, et la fonction filtrée est tracée en rouge.
\end{center}
On observe que la fonction filtrée correspond bien a la sinusoidale de période 1 et d'amplitude 1 qui compose la fonction originale. \newline

Si on filtre toutes les fréquences < 5 Hz on supprime le premier pic et on trouve la fonction suivante après ifft.
\begin{center}
\includegraphics[scale=0.6]{"fhathigh.png"} \newline
\end{center}
\begin{center}
\includegraphics[scale=0.6]{"f_and_fhathigh.png"} \newline
La fonction originale est tracée en bleu, et la fonction filtrée est tracée en rouge.
\end{center}
On observe que la fonction filtrée correspond a la sinusoidale de période 0.1 et d'amplitude 0.9 qui compose la fonction originale \newline

\subsection{MyData}
La fonction h(t) donne ce graphique:
\begin{center}
\includegraphics[scale=0.6]{"h_raw.png"} \newline
\end{center}
Sa transformée de Fourier est la suivante:
\begin{center}
\includegraphics[scale=0.6]{"hhat_raw.png"} \newline
\end{center}

Après filtrage la transformée ressemble a ceci:
\begin{center}
\includegraphics[scale=0.6]{"hhat_filtre.png"} \newline
\end{center}
Ce qui donne la fonction suivante
\begin{center}
\includegraphics[scale=0.6]{"h_filtre.png"} \newline
\end{center}

On constate que $h(t) \approx f(t)$
\section{Fonction de filtrage}

\section{Affichage}

\section{Script Principal}

\end{document}
